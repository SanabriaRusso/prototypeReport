\documentclass[a4paper]{journal}
\usepackage{listings}
\usepackage[table]{xcolor}

\lstset{ %
  %backgroundcolor=\color{white},   % choose the background color; you must add \usepackage{color} or \usepackage{xcolor}
  basicstyle=\footnotesize,        % the size of the fonts that are used for the code
  %breakatwhitespace=false,         % sets if automatic breaks should only happen at whitespace
  %breaklines=true,                 % sets automatic line breaking
  captionpos=b,                    % sets the caption-position to bottom
  %commentstyle=\color{mygreen},    % comment style
  %deletekeywords={...},            % if you want to delete keywords from the given language
  %escapeinside={\%@}{@)},          % if you want to add LaTeX within your code
  %extendedchars=true,              % lets you use non-ASCII characters; for 8-bits encodings only, does not work with UTF-8
  frame=single,                    % adds a frame around the code
  keepspaces=true,                 % keeps spaces in text, useful for keeping indentation of code (possibly needs columns=flexible)
  %keywordstyle=\color{blue},       % keyword style
  %language=Assembler,                 % the language of the code
  %morekeywords={*,...},            % if you want to add more keywords to the set
  numbers=left,                    % where to put the line-numbers; possible values are (none, left, right)
  numbersep=5pt,                   % how far the line-numbers are from the code
  %numberstyle=\tiny\color{mygray}, % the style that is used for the line-numbers
  %rulecolor=\color{black},         % if not set, the frame-color may be changed on line-breaks within not-black text (e.g. comments (green here))
  %showspaces=false,                % show spaces everywhere adding particular underscores; it overrides 'showstringspaces'
  %showstringspaces=false,          % underline spaces within strings only
  %showtabs=false,                  % show tabs within strings adding particular underscores
  %stepnumber=2,                    % the step between two line-numbers. If it's 1, each line will be numbered
  stringstyle=\color{mymauve},     % string literal style
  tabsize=2,                       % sets default tabsize to 2 spaces
  title=\lstname                   % show the filename of files included with \lstinputlisting; also try caption instead of title
}

\begin{document}

\title{Report: testing CSMA/ECA implementation}
\author{Luis Sanabria-Russo}
\date{\today}
\maketitle

\begin{abstract}
We have performed many tests with the CSMA/ECA prototype. Although we were unable to find a clean environment that would remove the possibilities of external noise, tests revealed the implementation of the deterministic backoff after a successful transmission, which reduced the number of retransmitted frames. This report aims at gathering all the implementation details in order to be to minimize errors while repeating the tests, and also help us understand the behaviour or the implementation.
\end{abstract}

\section{OpenFWWF: modifying the \texttt{set\_backoff\_time} function}

You may find the code of the backoff function in Listing~\ref{backoffFunction}.

\lstinputlisting[language={[x86masm]Assembler}, caption=\texttt{set\_backoff\_time} function, escapeinside={@}{@}, label=backoffFunction]{"code/backoff.asm"}

The modifications we made based on our current understanding of the code are:

\begin{enumerate}
	\item \underline{Line \ref{checkWindow}}: we check if the current contention window is equal to the minimum contention window, if so it will \emph{jump} to the {\texttt{deterministic\_backoff}} branch. This is true when:
	\begin{itemize}
		\item the machine powers on.
		\item a packet is discarded due to reaching the retransmission limit.
		\item After a successful transmission.
	\end{itemize}
	\item \underline{Line \ref{keepDCF}}: if the current contention window is not the minimum, then the execution flow will jump to the {\texttt {continue\_backoff\_calculations}} branch.
	\item \underline{Line \ref{setBackoff}}: the deterministic backoff is stored in the General Purpose Register 5 ({\texttt{GP\_REG5}}), which is the one used for the backoff operations prior storing its value in memory (in Line~\ref{storeBackoff}, this is the default operation, we did not add this line). In this example the value {\texttt{0x100}} is stored in {\texttt{GP\_REG5}}.
\end{enumerate}

Apart from the modifications specified above, we also changed the minimum contention window to be two times the tested deterministic backoff, that is, if the backoff is $B_{d}=16$ slots, then the minimum contention window is set to $CW_{\min}=32$, and so on.

\section{The setup}
As usual, we performed the test with an Access Point (AP), a wired server and the Alix 2d2 as a host. The access point broadcasts a network that has MAC address access restrictions to prevent devices not in the test from joining in.

The test environment is not as clean as we would like to be. Although we performed channel scans and set up the AP to the WiFi channel with less intruders, it did not prevent external transmissions from causing collisions in our experiment.

\section{File transfers using {\texttt{scp}}}
We wanted to know approximately how long it takes CSMA/ECA stations to transmit $20$~MB of data (see Table~\ref{tab:beca-transfer}) to the wired server, this way we might identify faulty cards. Although the tool might not be appropriate for a complete analysis because of the involvement of higher layers congestion control (TCP), the Secure Copy ({\texttt{scp}}) command may serve as an initial pointer. It is not considered a reference metric, though. We are still looking for the appropriate tool.

\begin{table}[htbp]
		\centering
		\caption{CSMA/ECA nodes transfering $20$~MB of data with no other contenders}
		\label{tab:beca-transfer}
		\begin{tabular}{|c|c|c|c|c|}
			\hline
			\cellcolor{black} & \multicolumn{4}{|c|}{{\bfseries Avg. Tx Time (s)}}\\
			\hline
			{\bfseries ($B_{d}$, $CW_{\min}$)} & {\bfseries Node 1} & {\bfseries Node 2} & {\bfseries Node 3} & {\bfseries Node 4}\\
			\hline
			(31, 63) & $8$ & $9$ & $10.2$ & $9.8$\\
			\hline
		\end{tabular}
\end{table}

The data in Table~\ref{tab:beca-transfer} suggests that cards are able to perform similarly.

\subsection{Two simultaneous contenders}
CSMA/ECA's enhancements are related to a better collision avoidance, thus trying with only one station might in fact result in a lower throughput due to the more aggressive backoff mechanism in CSMA/CA. 

Table~\ref{tab:beca-transfer-couple} shows the average time it took to transfer $20$~MB of data in a network with two CSMA/ECA contenders.

\begin{table}[htbp]
		\centering
		\caption{Two CSMA/ECA nodes transfering $20$~MB of data}
		\label{tab:beca-transfer-couple}
		\begin{tabular}{|c|c|c|}
			\hline
			\cellcolor{black} & \multicolumn{2}{|c|}{{\bfseries Avg. Tx Time (s)}}\\
			\hline
			{\bfseries ($B_{d}$, $CW_{\min}$)} & {\bfseries Node 1} & {\bfseries Node 2}\\
			\hline
			(255, 511) & $17.8$ & $17.6$\\
			\hline
		\end{tabular}
\end{table} 

\bibliographystyle{journal}
\bibliography{../ref}

\end{document}