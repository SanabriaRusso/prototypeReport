\documentclass[a4paper]{journal}
\usepackage{listings}
\usepackage[table]{xcolor}
\usepackage{graphicx}

\lstset{ %
  %backgroundcolor=\color{white},   % choose the background color; you must add \usepackage{color} or \usepackage{xcolor}
  basicstyle=\footnotesize,        % the size of the fonts that are used for the code
  %breakatwhitespace=false,         % sets if automatic breaks should only happen at whitespace
  %breaklines=true,                 % sets automatic line breaking
  captionpos=b,                    % sets the caption-position to bottom
  %commentstyle=\color{mygreen},    % comment style
  %deletekeywords={...},            % if you want to delete keywords from the given language
  %escapeinside={\%@}{@)},          % if you want to add LaTeX within your code
  %extendedchars=true,              % lets you use non-ASCII characters; for 8-bits encodings only, does not work with UTF-8
  frame=single,                    % adds a frame around the code
  keepspaces=true,                 % keeps spaces in text, useful for keeping indentation of code (possibly needs columns=flexible)
  %keywordstyle=\color{blue},       % keyword style
  %language=Assembler,                 % the language of the code
  %morekeywords={*,...},            % if you want to add more keywords to the set
  numbers=left,                    % where to put the line-numbers; possible values are (none, left, right)
  numbersep=5pt,                   % how far the line-numbers are from the code
  %numberstyle=\tiny\color{mygray}, % the style that is used for the line-numbers
  %rulecolor=\color{black},         % if not set, the frame-color may be changed on line-breaks within not-black text (e.g. comments (green here))
  %showspaces=false,                % show spaces everywhere adding particular underscores; it overrides 'showstringspaces'
  %showstringspaces=false,          % underline spaces within strings only
  %showtabs=false,                  % show tabs within strings adding particular underscores
  %stepnumber=2,                    % the step between two line-numbers. If it's 1, each line will be numbered
  stringstyle=\color{mymauve},     % string literal style
  tabsize=2,                       % sets default tabsize to 2 spaces
  title=\lstname                   % show the filename of files included with \lstinputlisting; also try caption instead of title
}

\begin{document}

\title{Report: testing CSMA/ECA implementation}
\author{Luis Sanabria-Russo}
\date{\today}
\maketitle

\begin{abstract}
In order to avoid unnecesary repetitions of tests and in the hopes of finding clear answers to our questions, this report gathers results for CSMA/ECA tests made at using the reserved WiFi channel $14$. Results highlight the steep difference between the number of retransmitted frames seen with our prototype to be later compared against what it is expected according to analytical models.
\end{abstract}

\section{Introduction}
This document is a work-in-progress, it does not matter when you read it.

In order the identify the different characteristics of the implementation of CSMA/ECA using OpenFWWF, we first sniff the traffic generated by CSMA/CA stations. Then, we implement CSMA/ECA and contrast the results.

\subsection{The setup}
As usual, we performed the test with an Access Point (AP), a wired server and the Alix 2d2 as a host. The access point broadcasts a network that has MAC address access restrictions to prevent devices not in the test from joining in.

The test environment is not as clean as we would like to be. Although we performed channel scans and set up the AP to the WiFi channel with less intruders, it did not prevent external transmissions from causing collisions in our experiment.

If not mentioned otherwise, there are always four stations transmitting UDP segments at a fixed rate of 1 Mbps to a wired iPerf server. All traffic from the stations is captured using Wireshark and processed later with a custom made parser~\cite{pcapParser}. At the moment, we use the retransmission flag and the transmission time of each frame to compute throughput, average time of arrival and number of retransmitted frames for each host.
%
%\section{Distributed Coordination Function (DCF)}
%The $b43$ driver for wireless cards is the one installed in the Alix, thus we consider it the default firmware.
%
%Figure~\ref{DCF:default} shows the number of retransmitted frames as well as the average throughput per station when we vary the minimum contention window starting from the default setting ($CW_{\min}=31$).
%
%	\begin{figure}[htbp]
%	\centering
%		\includegraphics[width=0.74\linewidth,angle=-90]{figures/retransmissions-DCF.eps}
%		\caption{Percentage of retransmitted frames using DCF}
%		\label{DCF:default}
%	\end{figure}

\section{OpenFWWF: modifying the \texttt{set\_backoff\_time} function}

You may find the code of the backoff function in Listing~\ref{backoffFunction}.

\lstinputlisting[language={[x86masm]Assembler}, caption=\texttt{set\_backoff\_time} function, escapeinside={@}{@}, label=backoffFunction]{"code/backoff.asm"}

The modifications we made based on our current understanding of the code are:

\begin{enumerate}
	\item \underline{Line \ref{checkWindow}}: we check if the current contention window is equal to the minimum contention window, if so it will \emph{jump} to the {\texttt{deterministic\_backoff}} branch. This is true when:
	\begin{itemize}
		\item the machine powers on.
		\item a packet is discarded due to reaching the retransmission limit.
		\item After a successful transmission.
	\end{itemize}
	\item \underline{Line \ref{keepDCF}}: if the current contention window is not the minimum, then the execution flow will jump to the {\texttt {continue\_backoff\_calculations}} branch.
	\item \underline{Line \ref{setBackoff}}: the deterministic backoff is stored in the General Purpose Register 5 ({\texttt{GP\_REG5}}), which is the one used for the backoff operations prior storing its value in memory (in Line~\ref{storeBackoff}, this is the default operation, we did not add this line). In this example the value {\texttt{0x100}} is stored in {\texttt{GP\_REG5}}.
\end{enumerate}

Apart from the modifications specified above, in some scenarios we also changed the minimum contention window to be two times the tested deterministic backoff, that is, if the backoff is $B_{d}=16$ slots, then the minimum contention window is set to $CW_{\min}=32$, and so on.

%\section{File transfers using {\texttt{scp}}}
%We wanted to know approximately how long it takes CSMA/ECA stations to transmit $20$~MB of data (see Table~\ref{tab:beca-transfer}) to the wired server, this way we might identify faulty cards. Although the tool might not be appropriate for a complete analysis because of the involvement of higher layers congestion control (TCP), the Secure Copy ({\texttt{scp}}) command may serve as an initial pointer. It is not considered a reference metric, though. We are still looking for the appropriate tool.
%
%\begin{table}[htbp]
%		\centering
%		\caption{CSMA/ECA nodes transfering $20$~MB of data with no other contenders}
%		\label{tab:beca-transfer}
%		\begin{tabular}{|c|c|c|c|c|}
%			\hline
%			\cellcolor{black} & \multicolumn{4}{|c|}{{\bfseries Avg. Tx Time (s)}}\\
%			\hline
%			{\bfseries ($B_{d}$, $CW_{\min}$)} & {\bfseries Node 1} & {\bfseries Node 2} & {\bfseries Node 3} & {\bfseries Node 4}\\
%			\hline
%			(31, 63) & $8$ & $9$ & $10.2$ & $9.8$\\
%			\hline
%		\end{tabular}
%\end{table}
%
%The data in Table~\ref{tab:beca-transfer} suggests that cards are able to perform similarly.
%
%\subsection{Two simultaneous contenders}
%CSMA/ECA's enhancements are related to a better collision avoidance, thus trying with only one station might in fact result in a lower throughput due to the more aggressive backoff mechanism in CSMA/CA. 
%
%Table~\ref{tab:beca-transfer-couple} shows the average time it took to transfer $20$~MB of data in a network with two CSMA/ECA contenders.
%
%\begin{table}[htbp]
%		\centering
%		\caption{Two CSMA/ECA nodes transfering $20$~MB of data}
%		\label{tab:beca-transfer-couple}
%		\begin{tabular}{|c|c|c|}
%			\hline
%			\cellcolor{black} & \multicolumn{2}{|c|}{{\bfseries Avg. Tx Time (s)}}\\
%			\hline
%			{\bfseries ($B_{d}$, $CW_{\min}$)} & {\bfseries Node 1} & {\bfseries Node 2}\\
%			\hline
%			(255, 511) & $17.8$ & $17.6$\\
%			\hline
%		\end{tabular}
%\end{table} 

\section{Effects of the deterministic backoff}
According to the simulations performed with CSMA/ECA, after a transitory phase nodes achieve a collision-free schedule due to the deterministic backoff after successful transmissions. Nevertheless, in the real implementation we see retransmitted frames at all stages of the test. Figure~\ref{retransmissions} shows the approximate number of retransmitted frames sent to the wired server while performing an \texttt{iperf} test.

	\begin{figure}[htbp]
	\centering
		\includegraphics[width=0.7\linewidth,angle=-90]{figures/Ch14/retransmissions-bar.eps}
		\caption{Percentage of retransmitted frames}
		\label{retransmissions}
	\end{figure}

In the figure, when using CSMA/ECA the deterministic backoff is half the minimum contention window. That is, for BECA15: $B_d=15,~CW_{\min}=31$ and so on. The number besides OpenFWWF determines the minimum contention window value; for OpenFWWF63: $CW_{\min}=63$.

Figure~\ref{retransmissions} shows how the average number of retransmitted frames of BECA15 is lower than the average of OpenFWWF. One would expect to see a reduction of the fraction of retransmitted frames when doubling $CW_{\min}$ and $B_d$; nevertheless this does not happen.

After doing various tests, we have noticed that whenever the value of $CW_{\min}$ is changed, BECA starts behaving as DCF. The reason why this happens is still unknown.

	\subsection{Not changing $CW_{\min}$}
	To test our hypothesis, we repeated the tests changing only $B_d$ and fixing $CW_{\min}=31$, which is its default value. Figure~\ref{retransmissions-sameCW} shows the average number of retransmissions for different values of $B_d$ against OpenFWWF.
	
	\begin{figure}[htbp]
	\centering
		\includegraphics[width=0.7\linewidth,angle=-90]{figures/Ch14/retransmissions-sameCW.eps}
		\caption{Percentage of retransmitted frames fixing $CW_{\min}=31$}
		\label{retransmissions}
	\end{figure}
	
	As $B_d$ grows, the number of retransmitted packets is reduced given that some of the possible collisions are avoided due to the deterministic backoff. Two special cases, namely BECA127 and BECA255 experience a slight increase in the number of retransmitted packets. The reason is still unclear.
	
		\subsubsection{Just two stations}
		In the eve of the INFOCOM demo, we decided also to perform the tests with only two stations (the demo will consist on comparing 2 stations running BECA against 2 stations runnning DCF). Figure~\ref{retransmissions-sameCW-2stas} shows the results.
		
		We can see how the average fraction of retransmitted packets per station is similar in both Figure~\ref{retransmissions} and~\ref{retransmissions-sameCW-2stas} for the same protocol.
		
		\begin{figure}[h]
		\centering
			\includegraphics[width=0.7\linewidth,angle=-90]{figures/Ch14/retransmissions-sameCW-2stas.eps}
			\caption{Percentage of retransmitted frames fixing $CW_{\min}=31$ and only two contenders}
			\label{retransmissions-sameCW-2stas}
		\end{figure}
		
		The average throughput per station achieved with each tested protocol is shown in Figure~\ref{throughput-2stas}.
		\begin{figure}[h]
		\centering
			\includegraphics[width=0.7\linewidth,angle=-90]{figures/Ch14/throughput-bar-sameCW-2stas.eps}
			\caption{Average throughput per station}
			\label{throughput-2stas}
		\end{figure}

		
	
\bibliographystyle{IEEEtran}
\bibliography{../ref}

\end{document}